\documentclass[UTF8,a4paper,twoside]{article}
\usepackage[UTF8]{ctex}  % 加载ctex包,支持中文
\usepackage{pdfpages}
\usepackage{tocvsec2}

% 设置参考文献
% 引入natbib包,参考文献格式相关
\usepackage[sectionbib]{natbib}
\usepackage{chapterbib}						
% 引入chapterbib包,可以分章节显示参考文献,且参考文献编号各自独立
% 参考文献格式
\usepackage{gbt7714}
% 封面信息
\newcommand{\stuname}{张三}			 	  % 学生姓名
\newcommand{\stuid}{3210xxxxxx}		  		 % 学生学号
\newcommand{\teaname}{李四}		    	  % 指导教师
\newcommand{\stugrade}{2021级}			 		% 学生年级
\newcommand{\stumajor}{信息工程} 	 		% 学生专业
\newcommand{\stuclass}{信息工程2104班} 	 		% 学生班级
\newcommand{\stucollege}{信息与电子工程学院} 		% 学生所在学院
\newcommand{\stutitle}{毕业设计题目}		% 毕设题目
\newcommand{\stuengtitlelineone}{The First Line of Your English Title}			% 毕设英文题目
\newcommand{\stuengtitlelinetwo}{The Second Line of Your English Title}

% 页面设置
% A4纸张大小 上下左边边距参考Word中"适中"类型
% A4纸张宽21cm 高29.7cm
\usepackage{geometry}	
\geometry{left=1.91cm, right=1.91cm, top=2.54cm, bottom=2.54cm}
\usepackage{subfigure}  % 导入subfigure宏包
% 设置首行缩进2字符
% 使用 \noindent 命令可以取消缩进
\usepackage{indentfirst}
\setlength{\parindent}{2em}


% 字体设置
\usepackage{fontspec}
\usepackage{amsfonts}
\usepackage{amsmath,amssymb,bm}
\usepackage{upgreek}
\usepackage{pifont}
\setmainfont[Path=fonts/, BoldFont=timesbd.ttf]{times.ttf}					% 缺省英文字体为Times New Roman
\setCJKmainfont[Path=fonts/, BoldFont=SimFangBold.ttf]{SimFang.ttf}			% 缺省中文字体为 仿宋
\setCJKfamilyfont{fs}[Path=fonts/, BoldFont=SimFangBold.ttf]{SimFang.ttf}
\newcommand{\zhengwen}{\CJKfamily{fs}\zihao{-4}}	% 正文字体 仿宋小4号
\setCJKfamilyfont{Heiti}[Path=fonts/, BoldFont=SimHeiBold.ttf]{SimHei.ttf}
\newcommand{\cover}{\CJKfamily{fs}\zihao{3}}
\setCJKfamilyfont{Songti}[Path=fonts/, BoldFont=SimSunBold.ttf]{SimSun.ttc}
\setCJKfamilyfont{Lishu}[Path=fonts/, BoldFont=lishu.ttf]{lishu.ttf}
\newcommand{\header}{\CJKfamily{Songti}\zihao{-5}}
\newcommand{\imageortable}{\CJKfamily{Songti}\zihao{5}}

% 行距设置
\usepackage{setspace}
\linespread{1.625}\selectfont		% 1.5倍字号,这与word中的1.5倍行距有一点差别

% 多级标题设置
\usepackage{titlesec}
\setcounter{secnumdepth}{4}		% 设置标题层次共4层
\titleformat{\section}[block]{\zihao{3}\bfseries}{\chinese{section}、}{0pt}{}
\titleformat{\subsection}[block]{\zihao{3}\bfseries}{\arabic{subsection}}{0.5em}{}
\titleformat{\subsubsection}[block]{\zihao{-3}\bfseries}{\arabic{subsection}.\arabic{subsubsection}}{0.5em}{}
\titleformat{\paragraph}{\zihao{4}\bfseries}{\arabic{subsection}.\arabic{subsubsection}.\arabic{paragraph}}{0.5em}{}
% 设置段间距
\titlespacing{\section}{0pt}{12pt}{6pt}				% 标题1 段前12磅 段后6磅
\titlespacing{\subsection}{0pt}{13pt}{13pt}			% 标题2 段前13磅 段后13磅
\titlespacing{\subsubsection}{0pt}{13pt}{13pt}		% 标题3 段前13磅 段后13磅
\titlespacing{\paragraph}{0pt}{13pt}{13pt}			% 标题4 段前13磅 段后13磅

% 定义一个新的引用命令,自动将引用放到文本的上方
\newcommand{\upcite}[1]{\textsuperscript{\citep{#1}}}

% 插入图片宏包
% 多个浮动体连续排布用参数H进行固定,如下所示,不用H会出现难以预料的排布
% \begin{figure}[H]
% content...
% \end{figure}
\usepackage{graphicx}
\usepackage{subfigure}
\usepackage{caption}
\usepackage{float}
\captionsetup{labelsep=quad,labelfont=bf,font=singlespacing}
% \renewcommand{\thefigure}{\arabic{subsection}.\arabic{figure}}
% \renewcommand{\theequation}{\arabic{subsection}.\arabic{equation}}  % 公式的编号格式
% 插入表格宏包
\usepackage{booktabs}
\usepackage{longtable}
\usepackage{multirow}
\usepackage{array}
\renewcommand{\thetable}{\arabic{subsection}.\arabic{table}}
\usepackage{enumerate}

\usepackage{caption}

% 设置图表标题字体

\renewcommand{\figurename}{\imageortable\bfseries 图}
\renewcommand{\tablename}{\imageortable\bfseries 表}

% 设置目录
\usepackage{titletoc}
\setcounter{tocdepth}{3}
\renewcommand{\thesection}{\CJKfamily{Heiti}\chinese{section}、}
\renewcommand{\thesubsection}{\arabic{subsection}}
\renewcommand{\thesubsubsection}{\arabic{subsection}.\arabic{subsubsection}}
\renewcommand{\theparagraph}{\arabic{subsection}.\arabic{subsubsection}.\arabic{paragraph}}
\titlecontents{section}[2em]{\CJKfamily{Heiti}\bfseries\zihao{-4}}{\contentslabel{2em}}{}{\titlerule*[0.5pc]{$\cdot$}\contentspage}
\titlecontents{subsection}[2.5em]{\bfseries\zihao{-4}}{\contentslabel{1em}}{}{\titlerule*[0.5pc]{$\cdot$}\contentspage}
\titlecontents{subsubsection}[4.5em]{\zihao{-4}}{\contentslabel{1.83em}}{}{\titlerule*[0.5pc]{$\cdot$}\contentspage}
\titlecontents{paragraph}[5.5em]{\zihao{-4}}{\contentslabel{2.67em}}{}{\titlerule*[0.5pc]{$\cdot$}\contentspage}
% 超链接
% colorlinks=true 超链接以颜色表示 false 超链接以方框框出
% linkcolor 指定颜色
% CJKbookmarks 让链接支持中文
\usepackage[colorlinks=true,linkcolor=black,citecolor=black,CJKbookmarks=true]{hyperref}
\newcommand{\myeqref}[1]{式 (\ref{#1})}
\newcommand{\mytaref}[1]{表 \ref{#1}}
\newcommand{\myfiref}[1]{图 \ref{#1}}
% 修改 algorithm 环境的标题为“算法”
% 直接重新定义 algorithm 的标题为中文
% 设置页眉页脚
\usepackage{fancyhdr}
\pagestyle{fancy}
\fancypagestyle{Index}{
	\setcounter{page}{1}\pagenumbering{Roman}
	\fancyhead[LO]{}
	\fancyhead[RO]{\header \stutitle}
	\fancyhead[LE]{\header 浙江大学本科生毕业论文(设计)}
	\fancyhead[RE]{}
	\fancyfoot[C]{\zihao{-5} \thepage}
}
\fancypagestyle{Require}{
    \fancyhead[L]{}
	\fancyhead[R]{\header \stutitle}
    \fancyfoot[C]{} % 清空页脚的页码
}
\fancypagestyle{Content}{
	\setcounter{page}{1}\pagenumbering{arabic}
	\fancyhead[LO]{}
	\fancyhead[RO]{\header \stutitle}
	\fancyhead[LE]{\header 浙江大学本科生毕业论文(设计)}
	\fancyhead[RE]{}
	\fancyfoot[C]{\zihao{-5} \thepage}
}

% 插入公式
\usepackage{amsmath}
\usepackage{indentfirst}
\usepackage{algorithm}
\usepackage{algpseudocode}

\numberwithin{equation}{subsection} % 将公式编号与章节关联
\numberwithin{figure}{subsection} % 图的编号与章节关联
\numberwithin{table}{subsection}  % 表的编号与章节关联
\renewcommand{\contentsname}{\vspace{-\baselineskip}}											
% 这条命令可以控制引入编译的各子文件是否参与编译	

\begin{document}
    % 封面
    \pagestyle{empty}

\begin{center}
\vspace*{3ex}
\includegraphics[width=11cm]{images/zju.jpg} 

\vspace{1ex}
{\CJKfamily{Heiti}\zihao{-1}\bfseries
本\ \ 科\ \ 生\ \ 毕\ \ 业\ \ 论\ \ 文(设计)

\vspace{15pt}
中期检查报告}
  
 
\cover 
\vspace{32pt}
\makebox[1.6cm][l]{\textbf{题目}}\underline{\makebox[14.9cm]{\textbf{\stutitle}}}
\makebox[1.6cm][l]{}\underline{\makebox[14.9cm]{}}
\makebox[2.5cm][l]{\textbf{\zihao{4}研究方向}}
\underline{\makebox[14cm][l]{\textbf{1.\hspace{0.3cm}}}}\\
\makebox[2.5cm][l]{}
\underline{\makebox[14cm][l]{\textbf{2.\hspace{0.3cm}}}}  
\makebox[2.2cm][l]{\textbf{\zihao{4}关键词}}
\underline{\makebox[14.3cm][l]{}}\\
\makebox[2.2cm][l]{}
\underline{\makebox[14.3cm][l]{}}    

\vspace{50pt}
\makebox[3.2cm][l]{\bfseries 姓名与学号}\underline{\makebox[7cm]{\stuname\quad \stuid}}

\vspace{5pt}
\makebox[3.2cm][l]{\bfseries 指导教师}\underline{\makebox[7cm]{\teaname}}

\vspace{5pt}
\makebox[3.2cm][l]{\bfseries 专业班级}\underline{\makebox[7cm]{\stuclass}}

\vspace{5pt}
{\CJKfamily{Lishu}\zihao{-2}\textbf{信息与电子工程学院}}

\vspace{16ex}
\end{center}

    \pagestyle{Require}
    \pagestyle{empty}
\begin{center}
    \zihao{3} \CJKfamily{Songti}\textbf{毕业论文(设计)中期检查问答}
\end{center}
\begin{flushleft}
{\zihao{-4}\CJKfamily{Heiti}\textbf{学生填写部分:}}
\CJKfamily{Songti}\zihao{-4}
\begin{enumerate}
    \item 你和指导教师一般隔多长时间\underline{\makebox[1.5cm]{}}天面对面讨论,每次约 \underline{\makebox[1.5cm]{}}小时,地点在\underline{\makebox[4cm]{}}。截止目前,你与指导教师共进行课题讨论\underline{\makebox[1.5cm]{}}次(包含但不限于面对面讨论)。
    \item 你的毕业论文(设计)工作一般在\underline{\makebox[4cm]{}}地点进行,平均每周工作 \underline{\makebox[1.5cm]{}}小时。
    \item 你现在已完成的主要工作内容占到全部任务的\underline{\makebox[1.5cm]{}}\%,目前是否已经获得核心数据以支撑你的论文目标的达成\underline{\makebox[1.5cm]{}}(是/否)。\par
    \qquad 以目前的进展,你预计大约在\underline{\makebox[1.5cm]{}}月\underline{\makebox[1.5cm]{}}日左右可以实现论文的研究目标。 本次毕业论文(设计)要求在4月21日前写作完成并且查重通过,你能按时完成吗?\underline{\makebox[1.5cm]{}}如果不能,原因是\underline{\makebox[6cm]{}}。      
    \item 你是本届保研学生吗?\underline{\makebox[1.5cm]{}}(是/否);你是预评的优秀毕业生吗?\underline{\makebox[1.5cm]{}}(是/否)。(如“是”,请继续回答以下第5条对应问题)
    \item 你是否了解保研政策对毕设总评成绩的要求?是否了解优秀毕业生评选对毕设总评成绩的要求?达不到总评成绩要求的严重后果是什么?\\
    \underline{\makebox[\linewidth]{}}
\end{enumerate}
{\zihao{-4}\CJKfamily{Heiti}\textbf{指导老师填写部分:}}
\CJKfamily{Songti}\zihao{-4}
\begin{enumerate}
    \item 以上情况是否属实(是/否):\underline{\makebox[1.5cm]{}}\\
    如“否”,实际情况是:\underline{\makebox[6cm]{}}             
    \item 对学生毕设工作态度及工作成效是否满意(在以下选项中选择): \underline{\makebox[1.5cm]{}}        
    \begin{enumerate}[A.]
        \item 进度超前,给予表扬;
        \item 进度正常,希望努力;
        \item 进度滞后,给予警告;
        \item 几乎没有开展实际研究(设计)工作,导师决定放弃指导,终止毕设。
    \end{enumerate}
\end{enumerate}
{\hfill {\zihao{-4}指导老师签名:\underline{\makebox[3cm]{}}}}\\
\hfill 日期:\underline{\makebox[4.7cm]{{\zihao{-4}  2025年3 月\quad\quad  日}}}
\end{flushleft}

    \pagestyle{empty}
\begin{center}
    \zihao{-2} \textbf{本科生毕业论文(设计)任务书}
\end{center}
    \pagestyle{Content}
    \zihao{-4}
    \cleardoublepage
\begin{center}
    \zihao{-2} \textbf{中期报告}
\end{center}
    
\section{毕业论文(设计)研究目标}

引用\upcite{schweizer2013comparative}

\section{任务书及开题报告预定的内容及进度安排}

\section{目前已完成的设计(论文)工作及成果}

\section{存在问题和解决措施}

\section{后续需完成的设计(论文)工作及进度安排}

% 参考文献
\newpage
\bibliographystyle{gbt7714-numerical}
\phantomsection		% 要想目录中参考文献的超链接正确需要加这一语句
\section{参考文献}
{\normalfont\CJKfamily{Songti}\zihao{5}\setlength{\baselineskip}{14pt}
\renewcommand{\refname}{\vspace{-\baselineskip}}
\bibliography{reference/refs}}
	
			
\end{document}